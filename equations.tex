\documentclass[11point]{article}

\usepackage[letterpaper,margin=1in]{geometry}

\usepackage{amssymb}
\usepackage{amsmath}
\usepackage{bbm}

\newcommand{\beq}{\begin{eqnarray}}
\newcommand{\eeq}{\end{eqnarray}}
\renewcommand{\d}[1]{\:\mathrm{d}#1}
\renewcommand{\vec}[1]{\mathbf{#1}}

\newcommand{\one}{\mathbbm{1}}
\newcommand{\reals}{\mathbb{R}}

\newcommand{\df}[1]{\delta \left( #1 \right)}
\newcommand{\set}[1]{\left\{#1\right\}}

\newcommand{\var}[1]{\texttt{#1}}

\begin{document}
\section{Notation}
\begin{itemize}
\item
For any cell $i$ with center $\vec{x}_i$, let $R_i$ denote the 
region of space occupied by it. Assume that for any other cell 
$j$, that $R_i \cap R_j = \emptyset$. 
\item 
For any  computational 
mesh with voxels $\set{ \Omega }$ and corresponding 
volumes $\set{W}$, let $\rho( \Omega) $ denote the mean 
substrate density in voxel $\Omega$, and let 
$n(\Omega) = \int_\Omega \rho \d{V}$ denote the total
 amount of substrate in the voxel. 

Note that BioFVM tracks the mean substrate density in each voxel, 
so $\rho \equiv \rho( \Omega )$ throughout $\Omega$. 
\item 
For any voxel $\Omega_k$ with an index $k$, let $W_k$ denote 
its volume, define $\rho_k = \rho( \Omega_k )$, and 
define $n_k = n( \Omega_k)$. 

\item 
For any cell $i$ with center $\vec{x}_i$, let $\Omega_i$ denote the 
voxel containing cell $i$, with corresponding volume $W_i$. 
\item
Let $\one_i(\vec{x})$ be the characteristic  function for the cell, so that 
$\one_i(\vec{x}) =1$ inside the cell (inside $R_i$), and 
$\one_i(\vec{x}) = 0$ otherwise. 
\item 
Let $V_i = \int_{\reals^3} \one_i(\vec{x}) \d{V} = V_i$ be the total volume of cell $i$. 
\item 
For any cell $i$, let $N_i$ denote the \emph{internalized} total substrate. 

\end{itemize}

\section{Net extracellular substrate change 
due to the $i^\textrm{th}$ cell}
Note that in BioFVM the cells' contribution 
to changes in total substrate in any volume $\Omega$ is given by 
\beq
\frac{ \partial }{ \partial t} 
\int_\Omega \rho \d{V} & = &  
\sum_{\textrm{cells } i} 
\int_\Omega  \one_i(\vec{x})
\Bigl(  S_i \left( \rho^T_i - \rho \right)  - U_i \rho   \Bigr) \d{V}\\
& \approx & 
\sum_{\textrm{cells } i}
V_i \int_\Omega \df{ \vec{x} - \vec{x}_i }
\Bigl(  S_i \left( \rho^T_i - \rho \right)  - U_i \rho   \Bigr) \d{V}.
\eeq 

Now, let $\Omega = \Omega_i$ be the voxel containing $\vec{x}_i$ as 
defined above. Then assuming that only cell $i$ is in $\Omega_i$:  \beq
\frac{ d{n_i} }{ d{t} }  
= 
\frac{ \partial }{ \partial t} 
\int_{\Omega_i} \rho \d{V}
 & \approx &  
V_i 
\Bigl(  S_i \left( \rho^T_i - \rho(\vec{x}_i) \right)  - U_i \rho(\vec{x}_i)   \Bigr)  \\
& = & 
V_i 
\Bigl(  S_i \left( \rho^T_i - \rho_i \right)  - U_i \rho_i   \Bigr) .
\eeq 
(The case with multiple cells in a single computational voxel generalizes by performing this calculation separately for each cell contained in the voxel.)

Now, because $n_i = \rho_i W_i $, and asssuming $W_i$ is constant or changes very slowly compared to substrate densities, 
\beq
W_i \frac{ d{\rho_i} }{ d{t} }  & \approx &  
V_i 
\Bigl(  S_i \left( \rho^T_i - \rho_i \right)  - U_i \rho_i   \Bigr)   \\
\Longrightarrow 
\frac{d\rho_i }{dt} & \approx & 
\frac{ V_i }{W_i }
\Bigl( S_i \left( \rho_i^T - \rho_i \right) - U_i \rho_i \Bigr) 
\eeq 

\subsection{BioFVM implementation}
Now, let's apply a backward Euler scheme as in BioFVM, to determine the net change in total substrate in any time step with duration $\Delta t$: 
\beq
\frac{ \rho_i(t+\Delta t) - \rho_i(t)}{\Delta t} 
& \approx & 
\frac{ V_i }{W_i } 
\Bigl( S_i \left( \rho_i^T - \rho_i(t+\Delta t) \right) 
- U_i \rho_i( t + \Delta t ) \Bigr)\\
\Longrightarrow 
\rho_i( t+\Delta t) 
& \approx & 
\frac{\rho_i(t)  + c_1 }{c_2},
\eeq
where 
\beq
c_1 & = & \Delta t \frac{ V_i }{W_i } 
\left( S_i \rho_i^T \right) \\ 
c_2 & = & 1 + \Delta t \frac{ V_i }{W_i } 
\left( S_i + U_i \right) .
\eeq

This is the algorithm in 

\begin{center}\verb|void Basic_Agent::simulate_secretion_and_uptake( Microenvironment* pS, double dt )|
\end{center}

The constants $c_1$ and $c_2$ are set in 
\verb|void Basic_Agent::set_internal_uptake_constants( double dt )|. 

\subsection{Net extracellular substrate change}

Now, let's determine the change in total substrates in this implementation. First, 
\beq
n_i(t+\Delta t ) - n_i(t) & = & 
W_i \rho_i(t+\Delta t) - W_i \rho_i(t) \\
& = & 
W_i \left( \frac{\rho_i(t)+c_1}{c_2} - \rho_i(t) \right) \\ 
& = & 
W_i \left( \frac{\rho_i(t) + c_1  - c_2 \rho_i(t) }{c_2} \right) \\ 
& = & 
W_i \left( \frac{ (1-c_2) \rho_i(t) + c_1  }{c_2} \right) \\ 
\eeq
Notice that this can be calculated completely using constants that are already computed and used in BioFVM. 
\subsection{Algorithm}
We will use the following operations in the cell secretion/uptake function. (In the actual implementation, 
perform this on the entire vector of substrates, and use element-wise operations. i.e., Hadamard products and 
quotients.) 
\begin{enumerate}
\item 
\verb|change = 1 // 1|
\item 
\verb|change -= c2 // 1-c2|
\item 
\verb|change *= substrates // (1-c2)*rho|
\item 
\verb|change += c1 // (1-c2)*rho + c1|
\item 
\verb|change /= c2 // ((1-c2)*rho + c1)/c2|
\item 
\verb|change *= voxel_volume // W_i*((1-c2)*rho + c1)/c2|
\end{enumerate}
This is the net change in total substrates in $\Omega_i$. For conservation, 
the net chnage in cell $i$ is equal and opposite. Thus 
\begin{enumerate}
\setcounter{enumi}{6}
\item 
\verb|internalized_substrates -= change|
\end{enumerate}

\section{Additional option(s)}
If you set \verb|Basic_Agent::use_internal_densities_as_targets = true|, then whenever
the internal constants are changed, it sets 
\beq
\rho_i^* & = & \frac{N_i}{V_i}
\eeq
This criterion would be appropriate for non-active, diffusive secretion from the cell. 

Please note that 
if $\rho_i^* < \rho_i$, there is nothing in the mathematical form to prevent diffusion of the substrate 
back into the cell. If this is a concern, I suggest users manually test for that and set 
the secretion rates to zero accordingly. 

Future releases of PhysiCell may automate this testing, but we note that this test should be performed 
substrate-by-substrate. 




\end{document}